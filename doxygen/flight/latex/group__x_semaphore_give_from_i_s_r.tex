\hypertarget{group__x_semaphore_give_from_i_s_r}{\section{x\-Semaphore\-Give\-From\-I\-S\-R}
\label{group__x_semaphore_give_from_i_s_r}\index{x\-Semaphore\-Give\-From\-I\-S\-R@{x\-Semaphore\-Give\-From\-I\-S\-R}}
}
semphr. h 
\begin{DoxyPre}
 xSemaphoreGiveFromISR( 
                          xSemaphoreHandle xSemaphore, 
                          signed portBASE\_TYPE *pxHigherPriorityTaskWoken
                      )\end{DoxyPre}


{\itshape Macro} to release a semaphore. The semaphore must have previously been created with a call to v\-Semaphore\-Create\-Binary() or x\-Semaphore\-Create\-Counting().

Mutex type semaphores (those created using a call to x\-Semaphore\-Create\-Mutex()) must not be used with this macro.

This macro can be used from an I\-S\-R.


\begin{DoxyParams}{Parameters}
{\em x\-Semaphore} & A handle to the semaphore being released. This is the handle returned when the semaphore was created.\\
\hline
{\em px\-Higher\-Priority\-Task\-Woken} & x\-Semaphore\-Give\-From\-I\-S\-R() will set $\ast$px\-Higher\-Priority\-Task\-Woken to pd\-T\-R\-U\-E if giving the semaphore caused a task to unblock, and the unblocked task has a priority higher than the currently running task. If x\-Semaphore\-Give\-From\-I\-S\-R() sets this value to pd\-T\-R\-U\-E then a context switch should be requested before the interrupt is exited.\\
\hline
\end{DoxyParams}
\begin{DoxyReturn}{Returns}
pd\-T\-R\-U\-E if the semaphore was successfully given, otherwise err\-Q\-U\-E\-U\-E\-\_\-\-F\-U\-L\-L.
\end{DoxyReturn}
Example usage\-: 
\begin{DoxyPre}
 \#define LONG\_TIME 0xffff
 \#define TICKS\_TO\_WAIT 10
 xSemaphoreHandle xSemaphore = NULL;\end{DoxyPre}



\begin{DoxyPre}Repetitive task.
 void vATask( void * pvParameters )
 \{
    for( ;; )
    \{
We want this task to run every 10 ticks of a timer.  The semaphore 
was created before this task was started.\end{DoxyPre}



\begin{DoxyPre}Block waiting for the semaphore to become available.
        if( xSemaphoreTake( xSemaphore, LONG\_TIME ) == pdTRUE )
        \{
It is time to execute.\end{DoxyPre}



\begin{DoxyPre}...\end{DoxyPre}



\begin{DoxyPre}We have finished our task.  Return to the top of the loop where
we will block on the semaphore until it is time to execute 
again.  Note when using the semaphore for synchronisation with an
ISR in this manner there is no need to 'give' the semaphore back.
        \}
    \}
 \}\end{DoxyPre}



\begin{DoxyPre}Timer ISR
 void vTimerISR( void * pvParameters )
 \{
 static unsigned char ucLocalTickCount = 0;
 static signed portBASE\_TYPE xHigherPriorityTaskWoken;\end{DoxyPre}



\begin{DoxyPre}A timer tick has occurred.\end{DoxyPre}



\begin{DoxyPre}... Do other time functions.\end{DoxyPre}



\begin{DoxyPre}Is it time for vATask () to run?
        xHigherPriorityTaskWoken = pdFALSE;
    ucLocalTickCount++;
    if( ucLocalTickCount >= TICKS\_TO\_WAIT )
    \{
Unblock the task by releasing the semaphore.
        xSemaphoreGiveFromISR( xSemaphore, &xHigherPriorityTaskWoken );\end{DoxyPre}



\begin{DoxyPre}Reset the count so we release the semaphore again in 10 ticks time.
        ucLocalTickCount = 0;
    \}\end{DoxyPre}



\begin{DoxyPre}    if( xHigherPriorityTaskWoken != pdFALSE )
    \{
We can force a context switch here.  Context switching from an
ISR uses port specific syntax.  Check the demo task for your port
to find the syntax required.
    \}
 \}
 \end{DoxyPre}
 